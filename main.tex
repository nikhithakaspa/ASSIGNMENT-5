\documentclass[journal,12pt,twocolumn]{IEEEtran}

\usepackage{setspace}
\usepackage{gensymb}

\singlespacing


\usepackage[cmex10]{amsmath}

\usepackage{amsthm}

\usepackage{mathrsfs}
\usepackage{txfonts}
\usepackage{stfloats}
\usepackage{bm}
\usepackage{cite}
\usepackage{cases}
\usepackage{subfig}

\usepackage{longtable}
\usepackage{multirow}

\usepackage{enumitem}
\usepackage{mathtools}
\usepackage{steinmetz}
\usepackage{tikz}
\usepackage{circuitikz}
\usepackage{verbatim}
\usepackage{tfrupee}
\usepackage[breaklinks=true]{hyperref}
\usepackage{graphicx}
\usepackage{tkz-euclide}
\usepackage{float}


\usetikzlibrary{calc,math}
\usepackage{listings}
    \usepackage{color}                                            %%
    \usepackage{array}                                            %%
    \usepackage{longtable}                                        %%
    \usepackage{calc}                                             %%
    \usepackage{multirow}                                         %%
    \usepackage{hhline}                                           %%
    \usepackage{ifthen}                                           %%
    \usepackage{lscape}     
\usepackage{multicol}
\usepackage{chngcntr}

\DeclareMathOperator*{\Res}{Res}

\renewcommand\thesection{\arabic{section}}
\renewcommand\thesubsection{\thesection.\arabic{subsection}}
\renewcommand\thesubsubsection{\thesubsection.\arabic{subsubsection}}

\renewcommand\thesectiondis{\arabic{section}}
\renewcommand\thesubsectiondis{\thesectiondis.\arabic{subsection}}
\renewcommand\thesubsubsectiondis{\thesubsectiondis.\arabic{subsubsection}}


\hyphenation{op-tical net-works semi-conduc-tor}
\def\inputGnumericTable{}                                 %%

\lstset{
%language=C,
frame=single, 
breaklines=true,
columns=fullflexible
}
\begin{document}


\newtheorem{theorem}{Theorem}[section]
\newtheorem{problem}{Problem}
\newtheorem{proposition}{Proposition}[section]
\newtheorem{lemma}{Lemma}[section]
\newtheorem{corollary}[theorem]{Corollary}
\newtheorem{example}{Example}[section]
\newtheorem{definition}[problem]{Definition}

\newcommand{\BEQA}{\begin{eqnarray}}
\newcommand{\EEQA}{\end{eqnarray}}
\newcommand{\define}{\stackrel{\triangle}{=}}
\bibliographystyle{IEEEtran}
\providecommand{\mbf}{\mathbf}
\providecommand{\pr}[1]{\ensuremath{\Pr\left(#1\right)}}
\providecommand{\qfunc}[1]{\ensuremath{Q\left(#1\right)}}
\providecommand{\sbrak}[1]{\ensuremath{{}\left[#1\right]}}
\providecommand{\lsbrak}[1]{\ensuremath{{}\left[#1\right.}}
\providecommand{\rsbrak}[1]{\ensuremath{{}\left.#1\right]}}
\providecommand{\brak}[1]{\ensuremath{\left(#1\right)}}
\providecommand{\lbrak}[1]{\ensuremath{\left(#1\right.}}
\providecommand{\rbrak}[1]{\ensuremath{\left.#1\right)}}
\providecommand{\cbrak}[1]{\ensuremath{\left\{#1\right\}}}
\providecommand{\lcbrak}[1]{\ensuremath{\left\{#1\right.}}
\providecommand{\rcbrak}[1]{\ensuremath{\left.#1\right\}}}
\theoremstyle{remark}
\newtheorem{rem}{Remark}
\newcommand{\sgn}{\mathop{\mathrm{sgn}}}
\providecommand{\abs}[1]{\left\vert#1\right\vert}
\providecommand{\res}[1]{\Res\displaylimits_{#1}} 
\providecommand{\norm}[1]{\left\lVert#1\right\rVert}
%\providecommand{\norm}[1]{\lVert#1\rVert}
\providecommand{\mtx}[1]{\mathbf{#1}}
\providecommand{\mean}[1]{E\left[ #1 \right]}
\providecommand{\fourier}{\overset{\mathcal{F}}{ \rightleftharpoons}}
%\providecommand{\hilbert}{\overset{\mathcal{H}}{ \rightleftharpoons}}
\providecommand{\system}{\overset{\mathcal{H}}{ \longleftrightarrow}}
	%\newcommand{\solution}[2]{\textbf{Solution:}{#1}}
\newcommand{\solution}{\noindent \textbf{Solution: }}
\newcommand{\cosec}{\,\text{cosec}\,}
\providecommand{\dec}[2]{\ensuremath{\overset{#1}{\underset{#2}{\gtrless}}}}
\newcommand{\myvec}[1]{\ensuremath{\begin{pmatrix}#1\end{pmatrix}}}
\newcommand{\mydet}[1]{\ensuremath{\begin{vmatrix}#1\end{vmatrix}}}
\numberwithin{equation}{subsection}
\makeatletter
\@addtoreset{figure}{problem}
\makeatother
\let\StandardTheFigure\thefigure
\let\vec\mathbf
\renewcommand{\thefigure}{\theproblem}
\def\putbox#1#2#3{\makebox[0in][l]{\makebox[#1][l]{}\raisebox{\baselineskip}[0in][0in]{\raisebox{#2}[0in][0in]{#3}}}}
     \def\rightbox#1{\makebox[0in][r]{#1}}
     \def\centbox#1{\makebox[0in]{#1}}
     \def\topbox#1{\raisebox{-\baselineskip}[0in][0in]{#1}}
     \def\midbox#1{\raisebox{-0.5\baselineskip}[0in][0in]{#1}}
\vspace{3cm}
\title{Assignment -5}
\author{K.NIKHITHA}
\maketitle
\newpage
\bigskip
\renewcommand{\thefigure}{\theenumi}
\renewcommand{\thetable}{\theenumi}
Download all python codes from 
\begin{lstlisting}
https://github.com/K.NIKHITHA/tree/main/Assignment-5/Codes
\end{lstlisting}
%
and latex-tikz codes from 
%
\begin{lstlisting}
https://github.com/K.NIKHITHA/tree/main/Assignment6
\end{lstlisting}
%
\section{Question No. 2.69(a)}
Find the coordinates of the focus, axis of the parabola, the equation of the directrix and the length of the latus rectum   $y^2$ = 12$x$ .
%
\section{solution}
Given parabola is 
\begin{align}
y^2 &= 12x
\\
\implies y^2 - 12x &= 0
\end{align}
Vector form of given parabola is
\begin{align}
\vec{x}^T\myvec{0 & 0 \\ 0 & 1}\vec{x} + 2\myvec{-6 & 0}\vec{x} + 0 &= 0 
\end{align}
$\therefore$
\begin{align}
 \vec{V} = \myvec{0 & 0 \\ 0 & 1} ,
 \vec{u} = \myvec{-6\\0} ,
 f = 0
\end{align}
$\because$
$|\vec{V}|$ = 0 and $\lambda_1$ = 0 i.e. it is in standard form
\\
$\therefore$
\begin{align}
\vec{P}=\vec{I} \implies \vec{p_1} = \myvec{1\\0}
\\
\eta = \vec{u}^T\vec{p_1} = -6
\end{align}
The vertex $\vec{c}$ is given by
\begin{align}
\myvec{-12 & 0\\0 & 0\\0 & 1}\vec{c} &= \myvec{0\\0\\0}
\\
\implies \vec{c} &= \myvec{0\\0}
\end{align}
The focal length $\beta$ is given by
\begin{align}
\beta = \frac{1}{4}\abs{\frac{2\eta}{\lambda_2}} = \frac{1}{4}\abs{\frac{-12}{1}}= 3
\end{align}
The focus $\vec{F}$ is given by
\begin{align}
\vec{F} &= \vec{c} + {\frac{-2\eta\myvec{1 & 0}}{4}}^{T} 
\\
\implies \vec{F} &= \myvec{0\\0} + \myvec{3\\0}
\\
\implies \vec{F} &= \myvec{3\\0}
\end{align}
Axis of parabola is given by
\begin{align}
k(\vec{V}\vec{c}+\vec{u})^{T}\vec{x} &= 0 \quad\brak{  k \in \mathbb{R}}
\\
\implies k\myvec{-6 & 0}\vec{x} &= 0
\\
\implies \myvec{0 & 1}\vec{x} &= 0
\end{align}
Directrix of parabola is given by
\begin{align}
(\vec{V}\vec{c}+\vec{u})^T(\vec{x} +\beta) + \vec{u}^T\vec{c} + \vec{f} &= 0
\\
\implies \myvec{-6 & 0}(\vec{x+3}) &= 0
\\
\implies \myvec{1 & 0}\vec{x} &= -3
\end{align}
Latus rectum of parabola is given by
\begin{align}
(\vec{V}\vec{c}+\vec{u})^T(\vec{x} -\beta) + \vec{u}^T\vec{c} + \vec{f} &= 0
\\
\implies \myvec{-6 & 0}(\vec{x-3}) &= 0
\\
\implies \myvec{1 & 0}\vec{x} &=3
\end{align}
Length of latus rectum $l$ is 
\begin{align}
l &= \norm{\beta(\vec{V}\vec{c}+\vec{u})^T}
\\
\implies l &= \norm{3\myvec{-6 & 0}}
\\
\implies l &= 18 
\end{align}
Plot of given parabola
\numberwithin{figure}{section}
\begin{figure}[!ht]
\centering
\includegraphics[width=\columnwidth]{FIG-5.png}
\caption{Parabola $y^2=12x$ }
\label{fig:parabola}	
\end{figure}



\end{document}
